



\chapter{Hello World}



Before one begins to program, one must decide where they will write their code.
In the early days of computer science, it was common to use punch cards that would be fed into computers.
While we no longer require the use of such machines, we still need some method to not only run a program, but we also need some method to write a program effectively.
Before we begin our journey, one must prepare with the proper equipment.
This will consist of us obtaining a Java compiler and development environment.


\section*{Downloading the Java Development Kit}
As one can reasonably assume, in order to run Java code, one must download Java.
Specifically, we must \url{https://jdk.java.net/} in order to download the Java Development Kit (JDK).
The JDK will include the compiler (as well as other programs) that we will need in order to compile any programs that we need.
Once you have downloaded the SDK, we will need to install it.
This will slightly differ based on your operating system so we will cover the installation for Windows and MacOS.


\subsection*{Windows}

\subsection*{MacOS}



\section*{The Use of an Integrated Development Environment}
Now that we have downloaded our compiler, we must download an application that will allow us to efficiently code on our computer.
Just as one would need to use a web browser to browse the web or use a video editor to edit a video, it's recommended that beginner programers use a program to create Java applications.
Just as one has the choice of web browser (Mozilla Firefox, Internet Edge, Google Chrome, etc) and one has a choice of video editor (Final Cut Pro, Adobe Premier Pro, Apple iMovie, etc) there are a multitude of options for coding.
Specifically, we want to download an \gls{gls-IDE} (IDE) which is an application that allows users to read, write, execute, and debug code.

In truth, writing code involves us writing a plain text file.
This means that you can create an entire Java program using only a text editor and a terminal to exectue their code.
Indeed, there are many professional programmers who refuse to use any software that's more complicated than the aforementioned tools.
While these purists will proclaim that only true, enlightened software devlopers shun the use of modern \gls{IDE}s, I would recommend that you pay these people no heed.
One must remember that these are \textbf{professional} programmers, meaning that they have years of knowledge and muscle-memory aiding them through their development cycles.
An \gls{IDE} usually has many features that will serve as great aides while you create applications.
These features can include syntax checking, auto-complete options, intelligent renaming, the ability to import libraries, dead code removal, and they may even suggest fixes to your code.
This will eliminate many (but not all) first-time frustrations and rookie mistakes that you will inevitably commit.

This being said, as annoying as these enthusiasts may seem, they do have raise some good arguments.
Over-reliance on an IDE can hinder one's understanding of underlying steps to compiling and executing Java programs.
You can't always rely on having these programs available on every you'll operate.
For instance, you may be forced to log onto an offsite server where you will only be allowed to use a terminal.
Also, many system administrators have restrictions enabled on their systems which limit the programs that one can install on a computer.
Because it's almost guaranteed that you'll run into this scenarios, you don't want to find yourself unable to program anything.
In order to remedy this likely possibility, I will teach you to compile and run your Java programs using the command line terminal.
I won't go into advance scripting features; those lessons will be reserved for system administrators.
Rather, I will give you enough knowledge to be able to navigate to a project, build it, and run it.



All of this begs the question of which \gls{IDE} should we use.
If you are currently enrolled in Computer Science course, there's a very good chance that your instructor has already chosen your IDE for you.
In this case, you should continue to use the \gls{IDE} outlined in your curriculum.
This will minimize the amount of time you spend learning to use an your IDE and maximize the time you spend programming.
For anyone else, I suggest you just pick one.
According to Stack Overflow's 2022 survey\url{https://survey.stackoverflow.co/2022#most-popular-technologies-new-collab-tools-learn}, the most popular Java \gls{IDE} for beginners are, in decreasing order of popularity, IntelliJ, Eclipse, and Netbeans.
I don't suggest you spend too much time researching which one to choose.
Choice paralysis will not only impede the commencement of your learning at the expense of nothing.
While there are many debates over which application you should use, we only need something simple for our purposes.
Plus, if you ever grow tired of your choice, you can always switch at a later time.




