
\usepackage{hyperref}


\chapter{Hello World}



Before one begins to program, one must decide where they will write their code.
In the early days of computer science, it was common to use punch cards that would be fed into computers.
While we no longer require the use of such machines, we still need some method to not only run a program, but we also need some method to write a program effectively.
Before we begin our journey, one must prepare with the proper equipment.
This will consist of us obtaining a Java compiler and development environment.


\section*{Downloading the Java Development Kit}
As one can reasonably assume, in order to run Java code, one must download Java.
Specifically, we must \url{https://jdk.java.net/} in order to download the Java Development Kit (JDK).
The JDK will include the compiler (as well as other programs) that we will need in order to compile any programs that we need.
Once you have downloaded the SDK, we will need to install it.
This will slightly differ based on your operating system so we will cover the installation for Windows and MacOS.


\subsection*{Windows}

\subsection*{MacOS}



\section*{Downloading BlueJ}
Now that we have downloaded our compiler, we must download an application that will allow us to efficiently code on our computer.
Just as one would need to use a web browser to browse the web or use a video editor to edit a video, it's recommended that beginner programers use a program to create Java applications.
Just as one has the choice of web browser (Mozilla Firefox, Internet Edge, Google Chrome, etc) and one has a choice of video editor (Final Cut Pro, Adobe Premier Pro, Apple iMovie, etc) there are a multitude of options for coding.
Specifically, we want to download an Integrated Development Environment (IDE) which is an application that allows users to write, execute, and debug code.

In truth, writing code involves us writing a plain text file.
This means that you can create an entire Java program using only a text editor and a terminal to exectue their code.
Indeed, there are many professional programmers who refuse to use any software that's more complicated than the aforementioned tools.
While these purists will proclaim that only true, enlightened software devlopers shun the use of modern IDEs, I would recommend that you pay these people no heed.




