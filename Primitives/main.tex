


\chapter{Primitives}


Although you may be excited that you just compiled and ran your first program (and you should be), you may notice a lack of explanation for what any of the code that you see.
It may also bring you little comfort to hear that you will not be able to understand all of the code until you get through a significant portion of this book.
Truth be told, it took me two years to understand every single word in a Hello World program.
Nevertheless, you don't actually need to understand everything that has been just typed in order to continue learning about Java.
It may seem counter-intuitive, but this is essentially how every computer scientist begins programming.
As a personal experience, I noticed that I learned the most when I copied down the example programs in the my textbook and then modified them to see what would happen.
Not only will I encourage you to emulate this behavior, but I will have the early exercises be solvable using modifications of the example programs I give in main text.
This strategy will not only allow you to begin writing Java programs without being slowed down by lengthy explanations of certain terms, but it will also prepare you for being able to read solutions written by others in different contexts and being able to modify them for your purposes.









