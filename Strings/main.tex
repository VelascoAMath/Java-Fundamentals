

\chapter{Strings}


Now that you've officially created your first program, we can now take a closer look at the code you've written.
Specifcally, the component that actually outputs the phrase, "Hello, World!".
System.out.println() is a built-in Java method which will output whatever is between the parentheses.
For our first program, we put "Hello, World" inside of this method.
Given that I said that System.out.println() will output \textit{whatever} is between the parentheses, you may naturally infer that we can place some other message inside of our print statement; you would be absolutely correct to make such an assessment.
Try replacing the world ``World'' with your name.
As an example, I'm going to replace ``World'' with my name, Alfredo, and run the program.
The code and output are shown in figure \ref{fig:hello_alfredo}.

The technical term for this is a \gls{String literal}.
To first understand this term, we must understand the term \gls{String}.
A String is a sequence of characters.
A \gls{character} is a single unit used to represent text.




